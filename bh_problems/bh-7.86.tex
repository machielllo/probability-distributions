\input{header.tex}
\setcounter{theorem}{85}
\begin{exercise}BH.7.86
The concepts discussed here are a standard part of the education of medical doctors, and in data science in general.
\begin{hint}
The challenge for you is to try to understand the mathematics behind these concepts.
Read the exercise a number of times. I found it quite difficult to capture the concepts in formulas. (I solved it once. After two weeks,  I tried to solve it again, and found it just as hard as the first time.) Once you have the model, the technical part itself is simple.
\end{hint}
\begin{solution}
a.
It is given that $\P{T\leq t\given D=1} = G(t)$ and $\P{T\leq t\given D=0} = H(t)$.
In general, by Theorem 5.3.1.i, we can associate a rv to a CDF $F$; we say that the CDF $F$ \emph{induces} a rv $X$.
So, in the present case,   $G$ induces the rv $T_1$, and $H$ induces $T_0$.
With this, the sensitivity of the threshold $t_{0}$ is $\P{T_1>t_0} = \bar G(t_{0}) = 1-G(t_0)$, and the specificity is $\P{T_1<t_0} = H(t_0)$. Write $\bar H(t) = 1- H(t)$.

The ROC plot is the curve $s\to (\bar H(s), \bar G(t))$ where $s\in [a, b]$ parametrizes this curve in the $x,y$ plane.
Note that when $s=a$, $\bar H(a), \bar G(a)) = (1,1)$, and for $s=b$, $(\bar G(b), \bar H(b)) = (0,0)$.
So, the ROC curve starts at the point $(1,1)$ and moves to $(0,0)$ when $s$ increases from $a$ to $b$.

What is the area under the ROC curve?
Consider a tiny change in the parametrisation from $s$ to $s + \d s$.
At $s$ the height of the ROC curve is approx $\bar G(s)$, and the increase of the ROC curve along the $x$-axis is $h(s) \d s$, where $h = H'$ is the density of $H$.
Hence, the total area under the ROC curve is $\int_{a}^{b} \bar G(s) h(s) \d s$.

Next, the joint density of $T_{0}$ and $T_{1}$ is $f_{T_{0}, T_{1}}(s, t) = h(s) g(t)$. Therefore,
\begin{align*}
  \P{T_{1} > T_{0}}
  &= \int_{a}^{b}\int_s^{b} h(s) g(t) \d t \d s = \int_a^{b} h(s) \bar G(s) \d s.
\end{align*}
Clearly, the two integrals are the same.

b. when $g=h$, then $G(t)=H(t)$. The ROC curve is a line with slope 1. The area under the curve is $1/2$, which should be the case because when $g=h$, the test does not say anything useful about $T_{0}$ or $T_{1}$.

In the second case, the curve moves up very steeply when $s$ moves down from $b$.
Then the area under the ROC curve is nearly one, and the test is very informative.
\end{solution}
\end{exercise}
\input{trailer.tex}
