\input{header}
\subsection{BH.9.6.1}
\label{sec:bh.9.6.1}

Read BH.9.6.1 first. We will redo the computations and check the results with simulation. We will also cover many other topics as we proceed.



\begin{exercise}
We need to model the distribution of an individual demand $X$.
As an example, we can use the beta distribution.
In the example, $\mu$ and $\sigma$ are assumed as given, but for the beta distribution we need an $a$ and a $b$.
One way is to solve  for $a$ and $b$ for given $\mu$ and $\sigma$.
However, we can also be lazy and let the computer do the work.
Here is the code to do that.

Explain how this code works. (You can use the relevant documentation on the web on how the function fsolve works.) Why do I use $\mu=1/2$ and $\sigma^{2}=1/12$ as test case? Are the $a$ and $b$ that comes out of the solver correct? What does the final check do?
\begin{minted}[]{python}
from scipy.stats import beta
from scipy.optimize import fsolve

mu, var = 1 / 2, 1 / 12  # why this test case?


def func(x):
    a, b = x[0], x[1]
    m = a / (a + b)
    v = m * (1 - m) / (a + b + 1)
    return [m - mu, v - var]


a, b = fsolve(func, [2, 2])
print(a, b)
X = beta(a, b)
print(X.stats(moments='mv')) # final check.
\end{minted}

\begin{minted}[]{R}
library(pracma)

mu = 1 / 2
var = 1 / 12  # why this test case?

func = function(x){
  a = x[1]
  b = x[2]
  m = a / (a + b)
  v = m * (1 - m) / (a + b + 1)
  return (c(m - mu, v - var))
}

output = fsolve(func, c(2,2)) #x
a = output$x[1]
b = output$x[2]
print(paste(a, b))

X_mean = a / (a + b)
X_var = (X_mean * (1-X_mean )) / (a + b + 1)
print(paste(X_mean, X_var)) # final check

\end{minted}

\end{exercise}

\begin{exercise}
What do you expect to get when you would choose $\mu=2$? If it fails, why is that?
\end{exercise}

\begin{exercise}
What do you expect to get when you would choose $\sigma > 1/2$? If it fails, why is that? Hint, when an rv has a support on $[0,1]$, why is the variance at most $1/4$?
\end{exercise}

\begin{remark}
When setting up simulations and choosing variables, make sure that the variables you choose actually make any sense.
\end{remark}

Now we know how to find $a$ and $b$ for given $\mu$ and $\sigma$, we henceforth assume that $a=1$ and $b=2$, as this is easier to work with.

\begin{exercise}
Now that we know how to simulate individual demands, we can compute the mean and the variance with the formulas of the book and compare this to simulation.

Explain the lines marked as `this' of the code. (The rest must be clear for you. If not, then you have to be seriously concerned about your understanding of programming.)

\begin{minted}[]{python}
import numpy as np
from scipy.stats import poisson, beta
from scipy.optimize import fsolve

np.random.seed(3)

labda = 3
N = poisson(labda)
# print(N.rvs(100).mean()) # why this line?

X = beta(a=1, b=2)
mu, var = X.stats(moments='mv')

days = 1000
demands = np.zeros(days)
for i in range(days):
    demands[i] = X.rvs(N.rvs()).sum()  # this

print(demands.mean(), demands.var())
print(labda * mu, var * labda + mu * mu * labda) # this
\end{minted}

\begin{minted}[]{R}
set.seed(3)

labda = 3
# print(mean(rpois(100,labda))) # why this line?

days = 1000
demands = rep(0, days)

a = 1
b = 2
mu = a / (a + b)
var = (mu * (1 - mu)) / (a + b + 1)

for (i in 1:days){
  demands[i] = sum(rbeta(rpois(1,labda), 1, 2)) # this
}

print(paste(mean(demands), var(demands)))
print(paste(labda * mu, var * labda + mu * mu * labda)) # this
\end{minted}
\end{exercise}

\begin{exercise}
Here is the code to compute the mean and variance with Adam and Eve's laws. Explain how it works. In particular:
\begin{enumerate}
\item explain the lines marked as `this';
\item what are the contents of the demand matrix;
\item what is the \texttt{P};
\item how are \texttt{EX}, \texttt{EV} and \texttt{VE}  computed;
\item Note in particular the difference between \texttt{EXn} and \texttt{EXN}, and similarly for the variance. Use BH.9.2.1 to explain why I use a small $n$ and a large $N$ in the naming of the variables.
\end{enumerate}

Remark: I realize that this is a hard piece of code, but your understanding of Eve's law will increase by a lot. Some subtle points about the code that you \emph{don't} have to explain:
\begin{enumerate}
\item Suppose the largest amount of customers (jobs) that we saw during the simulation is $7$. Then on any day we can have $0, \ldots, 7$ demands, i.e, 8 possible outcomes. Thus, per row in the demands matrix we need to have 8 columns.
\item The numpy function (mean() function in R) that computes the mean gives a warning when it is applied to an empty array. (Just try it to see how it works.) We prevent this by computing the histogram (i.e., the empirical PDF) for $n=0$ separately. The same reasoning applies to the case when idx has zero length.
\end{enumerate}


\begin{minted}[]{python}
import numpy as np
from scipy.stats import poisson, beta, rv_discrete

np.random.seed(3)

labda = 3
N = poisson(labda)

X = beta(a=1, b=2)
mu, var = X.stats(moments='mv')

days = 1000
n_jobs = N.rvs(days)
n_max = n_jobs.max() + 1  # see the comments
demands = np.zeros((days, n_max))
for day in range(days):
    n = n_jobs[day]
    demands[day, :n] = X.rvs(n)  # this


P = np.zeros(n_max)
EXn = np.zeros(n_max)
VXn = np.zeros(n_max)
P[0] = np.argwhere(n_jobs == 0).size / days
for n in range(1, n_max):
    idx = np.argwhere(n_jobs == n)
    if idx.size == 0:
        continue
    P[n] = idx.size / days  # this
    EXn[n] = demands[idx, :n].mean() * n  # this
    VXn[n] = demands[idx, :n].var() * n  # this

EXN = rv_discrete(values=(EXn, P))  # this

print("mean:", labda * mu, EXN.mean())

VXN = rv_discrete(values=(VXn, P))  # this
print("EV: ", var * labda, VXN.mean())

VE = EXN.var()
print("VE: ", mu * mu * labda, VE)
\end{minted}


\begin{minted}[]{R}
library(discreteRV)
set.seed(3)

labda = 3

a = 1
b = 2
mu = a / (a + b)
var = (mu * (1 - mu)) / (a + b + 1)

days = 1000
n_jobs = rpois(days,labda)
n_max = max(n_jobs) + 1 # see the comments
demands = matrix(0,days, n_max) # this
for (day in 1:days){
  n = n_jobs[day]
  if(n == 0) next
  demands[day, 1:n] = rbeta(n, 1, 2) # this
}

P = rep(0, n_max)
EXn = rep(0, n_max)
VXn = rep(0, n_max)
P[1] = length(n_jobs[which(n_jobs == 0)])/days
for(n in 1:(n_max-1)){
  idx = which(n_jobs == n)
  if (length(idx) == 0){
    next
  }
  P[n+1] = length(idx) / days # this
  EXn[n+1] = mean(demands[idx, c(1:n)]) * n  # this
  VXn[n+1] = var(as.vector(demands[idx, c(1:n)])) * n # this
}

EXN = RV(outcomes = EXn, probs = P) # this
print(paste("mean:", labda * mu, E(EXN) ))

VXN = RV(outcomes = VXn, probs = P) # this
print(paste("EV: ", var * labda, E(VXN)))

VE = V(EXN)
print(paste("VE: ", mu * mu * labda, VE))
\end{minted}

\end{exercise}

\input{trailer}
